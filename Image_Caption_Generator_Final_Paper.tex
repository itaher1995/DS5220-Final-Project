\documentclass[11.5pt]{article}
\usepackage[margin=1in]{geometry}
\usepackage{hyperref}



\title{Image Caption Generator}

\author{Ibrahim Taher, Forrest Hooton}

\date{}

\begin{document}

\maketitle

\abstract
Please provide a brief abstract of your project.

\vspace{2mm}
\section{Introduction}

1.1 Problem Statement

Our goal was to create a program that generated a \texit{generative} textual caption from a received image with no other input. More specifically, the textual caption should describe the context of the image (i.e. a man riding a horse), rather than the content of an image (i.e. a man, a horse).

That technical problem statement is as follows:
Develop an algorithm in which we can feed an image $p_i$ and generate a caption $c_i$

1.2 Project Relevance

The relevance of image-caption generation models is widespread for today's purposes. Take for instance, those who are visually impaired. If they have the ability to read words at a closer distance, then using this model will allow them to have a better understanding of the context of an image. Also, caption generation is important for all forms of media. Articles that are located in newspapers or on their respective websites are often associated with images depicting events in the article. Image caption generation will remove the need for human supervision and manual creation of captions.$^9$

1.3 Related Work
1.4 Dataset

\section{Technical Approach}
Please describe the techniques you have used in order to address the problem. Describe in detail the classification/regression/other techniques you have used in order to tackle the problem.

\section{Experimental Results}
Describe the datasets used for your experiments. Be precise in describing all information about the datasets, including, classes, number of samples per class, features used to represent data, and all pre/post processing of the datasets.\\
Describe the details about the implementation of each algorithm, e.g., how you perform training, validation, testing, values of the hyperparameters and your methods for hyperparameter tuning, training/validation/testing error on the dataset, and all useful plots/tables that help to better interpret your results and your work.

\section{Participants Contribution}
Please list the name of the participants. For each participant explain in details the role he/she played in the project: explain which methods was implemented by which member, which dataset was processed by which member, which experimental results were generated by which members, etc.

\vspace{10mm}
** Please do not change the size of the fonts.

** Please note that your submission must be at most 7 pages long.

\end{document}
